\documentclass{article}

\usepackage{tabularx}
\usepackage{booktabs}

\title{Problem Statement and Goals\\\progname}

\author{\authname}

\date{}

\input{../Comments}
%% Common Parts

\newcommand{\progname}{NeuroMap} % PUT YOUR PROGRAM NAME HERE
\newcommand{\authname}{\textbf {Yasmin Ahmadi}\\\\January 20, 2025} % AUTHOR NAMES                  

\usepackage{hyperref}
    \hypersetup{colorlinks=true, linkcolor=blue, citecolor=blue, filecolor=blue,
                urlcolor=blue, unicode=false}
    \urlstyle{same}
                                


\begin{document}

\maketitle

\begin{table}[hp]
\caption{Revision History} \label{TblRevisionHistory}
\begin{tabularx}{\textwidth}{llX}
\toprule
\textbf{Date} & \textbf{Developer(s)} & \textbf{Change}\\
\midrule
January 20, 2025 & Yasmin Ahmadi & Creation of Document\\


\bottomrule
\end{tabularx}
\end{table}

\section{Problem Statement}

Temporal Response Functions (TRFs) are essential in neuroscience because they enable researchers to understand how neural systems encode and process continuous stimuli over time. They provide a systematic approach to modeling the relationship between continuous stimulus features and corresponding brain responses. Currently, there are two commonly used algorithms for estimating TRFs: one implemented in MATLAB and another integrated into the Python-based Eelbrain toolkit. The goal of this project is to port the MATLAB-based algorithm into the Eelbrain toolkit.

\subsection{Problem}
This project addresses the problem of implementing a TRF algorithm from a different infrastructure into the Eelbrain toolkit.

\subsection{Inputs and Outputs}

The inputs to the program will be predictor variables, which represent the stimulus, and neural response data, which are measurements of brain activity.

The output of the program will be the TRF, which maps predictor variables to neural responses over time.

\subsection{Stakeholders}
The stakeholders of this project are neuroscientists, users of the Eelbrain toolkit, and researchers studying the brain.
\subsection{Environment}
The software will be designed to run on standard laptops or desktops with Windows, macOS, or Linux operating systems.

\section{Goals}
The goal of this project is to make the MATLAB-based TRF algorithm available in the open-source Python toolkit, Eelbrain, for MEG and EEG data analysis.
\section{Stretch Goals}
Extending beyond the original scope of the project, a potential stretch goal could involve comparing the two TRF algorithms that will become available in the same toolkit upon completion of the project.
\newpage{}


\end{document}
